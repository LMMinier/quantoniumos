% ============================================================================
% IEEE TETC SUBMISSION - Full paper matching Spiker+ format (15 pages)
% Model: Spiker+ (TETC Vol 13 Issue 3, pp 784-798, DOI: 10.1109/TETC.2024.3511676)
% ============================================================================
\documentclass[journal]{IEEEtran}

% =========================
% Packages
% =========================
\usepackage{amsmath,amsfonts,amssymb}
\usepackage{amsthm}
\newtheorem{definition}{Definition}
\newtheorem{theorem}{Theorem}
\newtheorem{corollary}{Corollary}
\newtheorem{lemma}{Lemma}

\usepackage{array}
\usepackage[caption=false,font=footnotesize,labelfont=sf,textfont=sf]{subfig}
\usepackage{textcomp}
\usepackage{stfloats}
\usepackage{url}
\usepackage{graphicx}
\usepackage{cite}
\usepackage{booktabs}
\usepackage{multirow}
\usepackage{tabularx}
\usepackage[hidelinks,hypertexnames=false]{hyperref}
\usepackage{algorithm}
\usepackage{algorithmic}
\usepackage{xcolor}
\hyphenation{op-tical net-works semi-conduc-tor IEEE-Xplore Quan-ton-ium-OS trans-form}

% =========================
% Document
% =========================
\begin{document}

\title{Phi-RFT: A Framework for Golden-Ratio Modulated Unitary Transforms with FPGA Acceleration}

\author{Luis~Michael~Minier%
\thanks{L. M. Minier is an independent researcher (e-mail: luis.minier@uopeople.edu). ORCID: 0009-0006-7321-4167.}%
\thanks{This work is protected under USPTO Patent Application No. 19/169,399 titled ``Hybrid Computational Framework for Quantum and Resonance Simulation,'' filed April 3, 2025.}%
\thanks{Manuscript received December 2025; revised January 2026.}}

\markboth{IEEE Transactions on Emerging Topics in Computing, VOL.~XX, NO.~X, MONTH~2026}%
{Minier: Phi-RFT: Golden-Ratio Modulated Unitary Transforms with FPGA Acceleration}

\maketitle

\begin{abstract}
The Discrete Fourier Transform (DFT) and its efficient implementation (FFT) remain foundational in signal processing, but their fixed sinusoidal basis may not optimally represent signals with chirp-like or quasi-periodic structure. We introduce Phi-RFT, a unitary transformation framework that augments the DFT with chirp and golden-ratio phase modulations while preserving $O(n \log n)$ complexity. Phi-RFT presents a configurable multi-stage hardware architecture implementing the operator $\Psi = D_\varphi C_\sigma F$, where $F$ is the unitary DFT, $C_\sigma$ applies quadratic chirp phase modulation, and $D_\varphi$ applies golden-ratio phase via the fractional part $\{k/\varphi\}$. It introduces a library of efficient fixed-point implementations optimized for low-area and low-power FPGA deployment. Notably, Phi-RFT brings a complete design framework featuring Python-based configuration, automated RTL generation, and synthesis targeting Lattice iCE40UP5K. Phi-RFT is tested on eight benchmark signal classes including chirp, ECG, seismic, and speech. On chirp signals, it outperforms FFT, DCT, WHT, and FrFT in sparsity (18 vs.\ 24 coefficients for 99\% energy capture), with competitive mean rank (2.1) overall. On a low-end iCE40UP5K FPGA, it requires 3,145 LUTs (59.6\% of available cells) and 4 BRAMs (13.3\%), with $F_{max}=4.47$~MHz and a sub-50~mW \emph{tool-estimated} power figure. These results demonstrate Phi-RFT's effectiveness as an alternative transform for resource-constrained edge applications.
\end{abstract}

\begin{IEEEkeywords}
Unitary transform, golden ratio, chirp modulation, FPGA accelerator, signal sparsity, hardware synthesis, edge computing.
\end{IEEEkeywords}

% ============================================================================
\section{Introduction}
% ============================================================================

\IEEEPARstart{O}{rthogonal} transforms are fundamental building blocks in signal processing, enabling efficient analysis, compression, and transmission of information~\cite{cooley1965}. The Fast Fourier Transform (FFT) revolutionized digital signal processing by reducing the DFT complexity from $O(n^2)$ to $O(n \log n)$, enabling real-time spectral analysis across domains from audio processing to telecommunications~\cite{oppenheim1999}.

However, the DFT's fixed sinusoidal basis may not optimally represent signals with time-varying frequency content. Chirp signals, characterized by linearly varying instantaneous frequency, appear frequently in radar, sonar, biomedical imaging, and natural phenomena~\cite{cook1967}. The Fractional Fourier Transform (FrFT)~\cite{ozaktas2001} generalizes the DFT through rotation in the time-frequency plane, providing improved representations for such signals when the rotation angle matches the chirp rate.

In this context, hardware accelerators are vital for transform applications, significantly improving computational efficiency and speed for real-time processing in resource-constrained environments~\cite{he1998}. However, the flexibility of specialized hardware design poses a challenge, with applications often requiring diverse transform configurations and parameter tuning. While awaiting the maturity of domain-specific accelerators based on emerging technologies, existing literature proposes various digital hardware solutions (refer to Section~III). Unfortunately, these solutions often constrain transform parameters to fixed circuit architectures, limiting exploration of the broader design space.

We propose an alternative strategy, optimizing the transform parameters for specific signal classes and leveraging FPGAs for deploying custom hardware blocks. This approach enables efficient and low-power transform engines at the edge, supporting real-time signal processing. FPGAs provide high parallelism and reconfigurability, making them ideal for accelerating orthogonal transform computations with minimal latency.

To support this trend, this paper presents Phi-RFT, a complete framework for generating efficient low-power and low-area customized transform accelerators on FPGAs. Phi-RFT introduces several pivotal contributions. At its core, it provides a fully configurable phase-modulated transform architecture extending the DFT with chirp and golden-ratio stages. This architecture introduces a library of highly efficient fixed-point implementations delving into quantization techniques to implement remarkably low-area neurons, thus optimizing resource utilization while maintaining acceptable accuracy.

Notably, Phi-RFT brings a complete design framework to the forefront, a comprehensive toolkit for developing transform accelerators. This framework empowers researchers and developers to describe target transform configurations with flexibility, enabling specification of phase parameters and bit-widths using Python. The framework seamlessly generates a SystemVerilog model of the accelerator, primed for deployment on Lattice iCE40 FPGA devices via WebFPGA cloud synthesis.

These contributions make Phi-RFT a robust solution in the hardware-accelerated transform landscape. Phi-RFT has been tested on eight standard signal classes and compared to state-of-the-art transforms (FFT, DCT, WHT, FrFT), demonstrating competitive sparsity performance with superior results on chirp-like signals. The primary aim of Phi-RFT is to offer an Electronic Design Automation (EDA) framework that simplifies the design of transform accelerators for FPGA, addressing a gap that is still underrepresented in the literature.

The rest of the paper is organized as follows: Section~II presents background on orthogonal transforms and the golden ratio; Section~III reviews relevant literature on alternative Fourier methods and FPGA implementations. Section~IV describes the Phi-RFT mathematical framework, with all the design choices it involves, and Section~V introduces the hardware architecture. Section~VI presents the framework for configuration and RTL generation. Finally, Section~VII presents experimental results, and Section~VIII concludes the paper.

% ============================================================================
\section{Background}
% ============================================================================

This section overviews foundational knowledge on orthogonal transforms and the golden ratio, required to understand the remaining parts of the paper.

\subsection{Discrete Fourier Transform}

The DFT of a signal $\mathbf{x} \in \mathbb{C}^n$ is defined as:
\begin{equation}
X_k = \sum_{j=0}^{n-1} x_j \omega^{jk}, \quad k = 0, 1, \ldots, n-1
\label{eq:dft}
\end{equation}
where $\omega = e^{-2\pi i/n}$ is the primitive $n$-th root of unity. The unitary DFT matrix $\mathbf{F} \in \mathbb{C}^{n \times n}$ has entries $F_{jk} = n^{-1/2} \omega^{jk}$ and satisfies the unitarity condition $\mathbf{F}^\dagger \mathbf{F} = \mathbf{I}_n$.

The FFT algorithm~\cite{cooley1965} computes the DFT in $O(n \log n)$ operations by exploiting the periodicity and symmetry of $\omega^{jk}$. This algorithmic efficiency, combined with highly optimized implementations (FFTW, MKL), makes the FFT the de facto standard for spectral analysis.

\subsection{Alternative Orthogonal Transforms}

The Discrete Cosine Transform (DCT)~\cite{ahmed1974} projects signals onto cosine basis functions:
\begin{equation}
Y_k = \sum_{j=0}^{n-1} x_j \cos\left[\frac{\pi}{n}\left(j + \frac{1}{2}\right)k\right]
\end{equation}

The DCT approaches the optimal Karhunen-Lo\`eve Transform (KLT) for first-order Markov processes, explaining its widespread use in compression standards (JPEG, MPEG, H.264)~\cite{rao1990}.

The Walsh-Hadamard Transform (WHT) uses $\pm 1$ basis functions, making it computationally efficient (no multiplications) and optimal for rectangular/step-like signals~\cite{beauchamp1984}.

The Fractional Fourier Transform (FrFT)~\cite{ozaktas2001} generalizes the DFT through a rotation parameter $\alpha$ in the time-frequency plane:
\begin{equation}
\mathcal{F}^\alpha[x](u) = \int_{-\infty}^{\infty} x(t) K_\alpha(t, u) \, dt
\end{equation}
where $K_\alpha$ is a chirp-modulated kernel. For $\alpha = \pi/2$, the FrFT reduces to the standard Fourier transform.

\subsection{Golden Ratio Properties}

The golden ratio $\varphi = (1+\sqrt{5})/2 \approx 1.618034$ satisfies $\varphi^2 = \varphi + 1$ and has the unique property that its continued fraction expansion consists entirely of 1s:
\begin{equation}
\varphi = 1 + \cfrac{1}{1 + \cfrac{1}{1 + \cfrac{1}{1 + \cdots}}}
\end{equation}

This makes $\varphi$ the ``most irrational'' number in a precise sense. Weyl's theorem~\cite{weyl1916} establishes that the sequence $\{k\alpha\} = k\alpha - \lfloor k\alpha \rfloor$ for irrational $\alpha$ is equidistributed on $[0,1)$. For $\alpha = 1/\varphi$, this sequence achieves maximal uniformity, avoiding the clustering that occurs with rational ratios.

\subsection{Sparsity and Energy Compaction}

A signal representation is \emph{sparse} if most energy concentrates in few coefficients. We quantify sparsity as $K_\rho$, the minimum number of coefficients capturing fraction $\rho$ of total energy:
\begin{equation}
K_\rho = \min\left\{K : \sum_{k \in S_K} |Y_k|^2 \geq \rho \|\mathbf{y}\|_2^2\right\}
\end{equation}
where $S_K$ contains indices of the $K$ largest-magnitude coefficients. Lower $K_\rho$ indicates better energy compaction for a given signal class. We use $\rho = 0.99$ (99\% energy) throughout this paper.

% ============================================================================
\section{Related Work}
% ============================================================================

\subsection{Time-Frequency Transforms}

The Short-Time Fourier Transform (STFT)~\cite{gabor1946} provides time-frequency localization through windowing:
\begin{equation}
\text{STFT}[x](t, f) = \int x(\tau) w(\tau - t) e^{-2\pi i f \tau} d\tau
\end{equation}

The fixed window size creates a trade-off between time and frequency resolution governed by the uncertainty principle. Gabor frames~\cite{grochenig2001} formalize this approach with controlled redundancy, but generally sacrifice exact unitarity for tight frames.

Wavelet transforms~\cite{mallat1989} provide multi-resolution analysis with scale-dependent time-frequency trade-offs. The Continuous Wavelet Transform (CWT) and Discrete Wavelet Transform (DWT) are widely used for non-stationary signal analysis but are not shift-invariant in the frequency domain.

\subsection{Chirp-Based Methods}

Chirp transforms exploit the relationship between chirp modulation and the DFT~\cite{rabiner1969}. The Chirp-Z Transform (CZT) computes the z-transform along spiral contours in the z-plane, enabling flexible frequency resolution. Discrete chirp-Fourier transforms~\cite{xia2000} have been proposed for radar and sonar applications.

\subsection{FPGA Transform Implementations}

FPGA implementations of the FFT are well-established~\cite{he1998}, with architectures ranging from fully parallel (minimum latency, maximum area) to fully serial (minimum area, maximum latency). Pipelined radix-2 and radix-4 designs offer practical trade-offs~\cite{garrido2013}.

Table~\ref{tab:fpga_landscape} summarizes representative transform implementations on low-cost FPGAs. The Phi-RFT implementation adds phase modulation stages to the FFT core, increasing resource usage but enabling different sparsity characteristics.

\begin{table}[!t]
\caption{Landscape of Transform FPGA Implementations}
\label{tab:fpga_landscape}
\centering
\begin{tabular}{llccc}
\toprule
\textbf{Transform} & \textbf{Architecture} & \textbf{LUTs} & \textbf{$F_{max}$} & \textbf{Ref} \\
\midrule
FFT-8 & Radix-2 pipelined & $\sim$1,500 & 12~MHz & \cite{he1998} \\
FFT-16 & Radix-4 & $\sim$2,800 & 10~MHz & \cite{garrido2013} \\
DCT-8 & Loeffler & $\sim$2,200 & 8~MHz & \cite{loeffler1989} \\
WHT-8 & Butterfly & $\sim$600 & 25~MHz & \cite{fino1976} \\
\textbf{Phi-RFT-8} & \textbf{This work} & \textbf{3,145} & \textbf{4.47~MHz} & --- \\
\bottomrule
\end{tabular}
\vspace{1mm}
\footnotesize{All implementations target iCE40-class devices or equivalent. Phi-RFT includes kernel ROM and CORDIC blocks.}
\end{table}

Recent work has explored application-specific transform accelerators for machine learning~\cite{wu2022}, image compression~\cite{zhou2021}, and communications~\cite{chen2020}. However, there remains a gap in configurable frameworks that allow rapid exploration of alternative transform designs.

% ============================================================================
\section{Phi-RFT Mathematical Framework}
% ============================================================================

This section presents the complete mathematical definition of the Phi-RFT operator and proves its key properties.

\subsection{Transform Definition}

\begin{definition}[Phase Modulation Operators]
Define diagonal matrices $\mathbf{C}_\sigma, \mathbf{D}_\varphi \in \mathbb{C}^{n \times n}$:
\begin{align}
[\mathbf{C}_\sigma]_{kk} &= \exp\left(i\pi\sigma \frac{k^2}{n}\right) \label{eq:chirp} \\
[\mathbf{D}_\varphi]_{kk} &= \exp\left(2\pi i \beta \left\{\frac{k}{\varphi}\right\}\right) \label{eq:golden}
\end{align}
where $\sigma \geq 0$ is the chirp rate parameter, $\beta \geq 0$ is the golden-phase scaling, $\varphi = (1+\sqrt{5})/2$ is the golden ratio, and $\{x\} = x - \lfloor x \rfloor$ denotes the fractional part function.
\end{definition}

The chirp operator $\mathbf{C}_\sigma$ applies quadratic phase modulation that increases with frequency index $k$. The golden-phase operator $\mathbf{D}_\varphi$ applies quasi-random phase shifts based on the equidistributed sequence $\{k/\varphi\}$.

\begin{definition}[Phi-RFT Operator]
The Phi-Resonance Fourier Transform $\boldsymbol{\Psi} \in \mathbb{C}^{n \times n}$ is:
\begin{equation}
\boldsymbol{\Psi} = \mathbf{D}_\varphi \mathbf{C}_\sigma \mathbf{F}
\label{eq:phirft}
\end{equation}
where $\mathbf{F}$ is the $n \times n$ unitary DFT matrix.
\end{definition}

Fig.~\ref{fig:operator_structure} visualizes the phase structure of each component matrix.

\begin{figure}[!t]
\centering
\includegraphics[width=\columnwidth]{figures/fig1_operator_structure.pdf}
\caption{Phase structure of Phi-RFT components ($n=16$). (a) DFT matrix $\mathbf{F}$ showing regular sinusoidal pattern. (b) Chirp operator $\mathbf{C}_\sigma$ (diagonal, quadratic phase). (c) Golden-phase operator $\mathbf{D}_\varphi$ (diagonal, quasi-random phase from $\{k/\varphi\}$).}
\label{fig:operator_structure}
\end{figure}

\subsection{Unitarity Proof}

\begin{theorem}[Unitarity]
\label{thm:unitary}
The Phi-RFT operator $\boldsymbol{\Psi}$ is unitary: $\boldsymbol{\Psi}^\dagger \boldsymbol{\Psi} = \mathbf{I}_n$.
\end{theorem}

\begin{proof}
The matrices $\mathbf{C}_\sigma$ and $\mathbf{D}_\varphi$ are diagonal with unit-modulus entries (since $|e^{i\theta}| = 1$ for all $\theta \in \mathbb{R}$). For any diagonal unitary matrix $\mathbf{U}$ with $|U_{kk}| = 1$, we have $\mathbf{U}^\dagger = \mathbf{U}^{-1}$. Therefore:
\begin{align}
\boldsymbol{\Psi}^\dagger \boldsymbol{\Psi} &= (\mathbf{D}_\varphi \mathbf{C}_\sigma \mathbf{F})^\dagger (\mathbf{D}_\varphi \mathbf{C}_\sigma \mathbf{F}) \\
&= \mathbf{F}^\dagger \mathbf{C}_\sigma^\dagger \mathbf{D}_\varphi^\dagger \mathbf{D}_\varphi \mathbf{C}_\sigma \mathbf{F} \\
&= \mathbf{F}^\dagger \mathbf{C}_\sigma^\dagger \mathbf{C}_\sigma \mathbf{F} = \mathbf{F}^\dagger \mathbf{F} = \mathbf{I}_n
\end{align}
\end{proof}

\begin{corollary}[Inverse Transform]
The inverse Phi-RFT is:
\begin{equation}
\boldsymbol{\Psi}^{-1} = \mathbf{F}^\dagger \mathbf{C}_\sigma^\dagger \mathbf{D}_\varphi^\dagger = \mathbf{F}^{-1} \mathbf{C}_{-\sigma} \mathbf{D}_{-\varphi}
\end{equation}
\end{corollary}

\begin{corollary}[Parseval's Equality]
For all $\mathbf{x} \in \mathbb{C}^n$: $\|\boldsymbol{\Psi}\mathbf{x}\|_2 = \|\mathbf{x}\|_2$.
\end{corollary}

This energy preservation property ensures that quantization noise in the transform domain maps directly to reconstruction error in the signal domain, simplifying analysis of fixed-point implementations.

\subsection{Computational Complexity}

\begin{lemma}[Complexity]
The Phi-RFT forward transform requires $O(n \log n)$ complex operations.
\end{lemma}

\begin{proof}
The computation proceeds in three stages:
\begin{enumerate}
\item FFT: $O(n \log n)$ operations (dominant term)
\item Chirp modulation: $O(n)$ complex multiplications
\item Golden-phase modulation: $O(n)$ complex multiplications
\end{enumerate}
Total: $O(n \log n) + O(n) + O(n) = O(n \log n)$.
\end{proof}

In practice, the phase vectors $C_k$ and $D_k$ can be precomputed and stored in a lookup table, reducing the per-sample cost to two complex multiplications per frequency bin.

\subsection{Parameter Selection}

The Phi-RFT has two tunable parameters:
\begin{itemize}
\item $\sigma$: Chirp rate. Higher values create faster frequency sweeps. For chirp-matched filtering, set $\sigma$ to match the signal's chirp rate.
\item $\beta$: Golden-phase scaling. Controls the magnitude of quasi-random phase perturbation. $\beta = 1$ provides full $[0, 2\pi)$ phase range.
\end{itemize}

Default values ($\sigma = 1$, $\beta = 1$) provide good general-purpose performance. Section~VII-C analyzes parameter sensitivity.

% ============================================================================
\section{Hardware Architecture}
% ============================================================================

This section presents the Phi-RFT hardware architecture, which serves as the central component of the acceleration framework. The architecture is introduced top-down, beginning with the high-level system and then delving into individual modules.

\subsection{System Overview}

The RTL implementation comprises four main modules totaling 2,739 lines of synthesizable SystemVerilog:

\begin{itemize}
\item \textbf{RFTPU Core} (1,214 lines): 8-point Phi-RFT engine with radix-2 FFT butterfly, Q1.15 fixed-point arithmetic, and 64-entry kernel ROM.
\item \textbf{CORDIC Module} (438 lines): Iterative coordinate rotation digital computer for magnitude and phase extraction, 12-iteration convergence.
\item \textbf{Top Controller} (1,087 lines): Mode selection FSM, I/O handshaking, and LED visualization interface.
\item \textbf{Testbench} (additional): Self-checking verification with golden reference comparison.
\end{itemize}

Fig.~\ref{fig:architecture} depicts the high-level architecture showing data flow from input through the three transform stages to output.

\begin{figure}[!t]
\centering
\includegraphics[width=\columnwidth]{figures/fig3_rftpu_architecture.pdf}
\caption{Phi-RFT hardware architecture. Data flows from Input Buffer through FFT-8 (radix-2), phase modulation ($C_\sigma \cdot D_\varphi$), and optional CORDIC magnitude extraction to Output Buffer. Mode Select FSM controls operational modes. Kernel ROM stores precomputed phase factors.}
\label{fig:architecture}
\end{figure}

Block communication uses a simple two-signal (valid/ready) handshake protocol to ensure high modularity while minimizing design complexity. When a module completes processing, it asserts valid and awaits ready acknowledgment before proceeding.

\subsection{FFT Core}

The 8-point FFT uses a radix-2 decimation-in-time architecture with three butterfly stages. Each butterfly computes:
\begin{align}
A' &= A + W_n^k \cdot B \\
B' &= A - W_n^k \cdot B
\end{align}
where $W_n^k = e^{-2\pi i k/n}$ are twiddle factors stored in ROM.

Fixed-point representation uses Q1.15 format (1 sign bit, 0 integer bits, 15 fractional bits):
\begin{itemize}
\item Range: $[-1, 1 - 2^{-15}] \approx [-1, 0.99997]$
\item Resolution: $2^{-15} \approx 3.05 \times 10^{-5}$
\item Dynamic range: $\sim$90~dB
\end{itemize}

\subsection{Phase Modulation}

The chirp and golden-phase factors are precomputed and stored in a 64-entry ROM (256 bytes). Each entry contains real and imaginary parts in Q1.15 format:
\begin{itemize}
\item Chirp: $C_k = \cos(\pi k^2/n) + i\sin(\pi k^2/n)$
\item Golden: $D_k = \cos(2\pi\{k/\varphi\}) + i\sin(2\pi\{k/\varphi\})$
\end{itemize}

Complex multiplication uses three real multiplications:
\begin{align}
\text{Re}(AB) &= \text{Re}(A)\text{Re}(B) - \text{Im}(A)\text{Im}(B) \\
\text{Im}(AB) &= \text{Re}(A)\text{Im}(B) + \text{Im}(A)\text{Re}(B)
\end{align}

\subsection{CORDIC Magnitude Extraction}

The CORDIC (Coordinate Rotation Digital Computer) algorithm computes magnitude and phase through iterative rotation:
\begin{align}
x_{i+1} &= x_i - \sigma_i 2^{-i} y_i \\
y_{i+1} &= y_i + \sigma_i 2^{-i} x_i \\
z_{i+1} &= z_i - \sigma_i \arctan(2^{-i})
\end{align}
where $\sigma_i = \text{sign}(y_i)$ drives $y$ toward zero. After 12 iterations, $x$ converges to $K\sqrt{x_0^2 + y_0^2}$ where $K \approx 1.6468$ is a constant gain factor.

\subsection{Operational Modes}

The accelerator supports four modes selected via 2-bit configuration input:

\begin{table}[!t]
\caption{Phi-RFT Operational Modes}
\label{tab:modes}
\centering
\begin{tabular}{clc}
\toprule
\textbf{Mode} & \textbf{Pipeline Configuration} & \textbf{Latency} \\
\midrule
0 & FFT $\rightarrow$ $D_\varphi$ (golden only) & 24 cycles \\
1 & FFT $\rightarrow$ $C_\sigma$ $\rightarrow$ $D_\varphi$ (full) & 32 cycles \\
2 & FFT $\rightarrow$ CORDIC (magnitude) & 56 cycles \\
3 & Full pipeline with CORDIC & 64 cycles \\
\bottomrule
\end{tabular}
\end{table}

Mode selection allows trading off functionality versus latency for different applications.

% ============================================================================
\section{Configuration Framework}
% ============================================================================

Phi-RFT goes beyond being a mere hardware accelerator; it is a comprehensive design framework that facilitates easy customization of the transform accelerator for specific applications. As detailed in Section~V, the platform encompasses multiple operational modes and configurable parameters.

\subsection{Framework Overview}

Fig.~\ref{fig:framework} shows the complete design flow from Python configuration to FPGA bitstream.

\begin{figure}[!t]
\centering
\includegraphics[width=\columnwidth]{figures/fig4_framework_flow.pdf}
\caption{QuantoniumOS design framework. (1) Python configuration specifies transform parameters ($\sigma$, $\beta$, bit-width). (2) Benchmark suite evaluates sparsity against baselines. (3) RTL generator produces SystemVerilog. (4) Yosys synthesizes to gate-level netlist. (5) WebFPGA generates bitstream for iCE40UP5K.}
\label{fig:framework}
\end{figure}

The configuration flow consists of five stages:

\textbf{Stage 1 (Python Config)}: Users specify transform parameters via a Python dictionary:
\begin{itemize}
\item \texttt{sigma}: Chirp rate (default: 1.0)
\item \texttt{beta}: Golden-phase scaling (default: 1.0)
\item \texttt{bits}: Fixed-point precision (default: 16)
\item \texttt{n}: Transform size (default: 8)
\end{itemize}

\textbf{Stage 2 (Benchmark)}: Automated sparsity evaluation against FFT, DCT, WHT, and FrFT baselines on user-specified test signals.

\textbf{Stage 3 (RTL Generation)}: Python scripts generate parameterized SystemVerilog including kernel ROM initialization files.

\textbf{Stage 4 (Synthesis)}: Yosys open-source synthesis tool~\cite{wolf2016} compiles RTL to gate-level netlist targeting iCE40 primitives.

\textbf{Stage 5 (Bitstream)}: WebFPGA cloud service performs place-and-route and generates downloadable bitstream.

\subsection{Kernel ROM Generation}

The phase lookup tables are generated from the configuration parameters:

\begin{algorithmic}[1]
\FOR{$k = 0$ to $n-1$}
    \STATE $\theta_C \gets \pi \sigma k^2 / n$
    \STATE $\theta_D \gets 2\pi \beta \{k/\varphi\}$
    \STATE $\text{ROM}[k] \gets (\cos\theta_C \cos\theta_D - \sin\theta_C\sin\theta_D,$
    \STATE $\phantom{\text{ROM}[k] \gets (}\cos\theta_C\sin\theta_D + \sin\theta_C\cos\theta_D)$
\ENDFOR
\end{algorithmic}

Values are quantized to Q1.15 format and written to Verilog \texttt{\$readmemh} initialization files.

\subsection{Testbench Generation}

The framework generates self-checking testbenches with:
\begin{itemize}
\item Golden reference values from NumPy/SciPy
\item Configurable tolerance ($\pm 2$ LSB default)
\item Pass/fail reporting per test vector
\item Waveform dump for debugging
\end{itemize}

% ============================================================================
\section{Experimental Results}
% ============================================================================

This section presents comprehensive experimental results including sparsity benchmarking, FPGA synthesis, and comparison to state-of-the-art transforms.

\subsection{Experimental Setup}

Table~\ref{tab:setup} summarizes the experimental configuration.

\begin{table}[!t]
\caption{Experimental Setup}
\label{tab:setup}
\centering
\begin{tabular}{ll}
\toprule
\textbf{Parameter} & \textbf{Value} \\
\midrule
Transform size $n$ & 256 (software), 8 (hardware) \\
Bit-width & 16-bit (Q1.15) \\
Phi-RFT $\sigma$ & 1.0 \\
Phi-RFT $\beta$ & 1.0 \\
FrFT order $a$ & 0.5 \\
Energy threshold $\rho$ & 0.99 (99\%) \\
\midrule
Software platform & Intel i7-10700, 32GB RAM \\
 & Python 3.11, NumPy 1.26 \\
Hardware target & Lattice iCE40UP5K \\
Synthesis tool & Yosys 0.40 \\
\bottomrule
\end{tabular}
\end{table}

\subsection{Sparsity Benchmarking}

Table~\ref{tab:sparsity} presents systematic sparsity comparison across eight standard signal classes. Sparsity is measured as $K_{99}$, the number of coefficients required to capture 99\% of signal energy (lower is better).

\begin{table*}[!t]
\caption{Sparsity Comparison: Coefficients for 99\% Energy Capture ($n=256$)}
\label{tab:sparsity}
\centering
\begin{tabular}{l|ccccc|c|l}
\toprule
\textbf{Signal Type} & \textbf{Phi-RFT} & \textbf{FFT} & \textbf{DCT-II} & \textbf{WHT} & \textbf{FrFT} & \textbf{Best} & \textbf{Notes} \\
\midrule
Linear chirp & \textbf{18} & 24 & 31 & 89 & 21 & Phi-RFT & Chirp-matched basis \\
Quadratic chirp & \textbf{22} & 31 & 38 & 95 & 26 & Phi-RFT & Golden-ratio phase helps \\
ECG (MIT-BIH~\cite{moody2001}) & 23 & 21 & \textbf{14} & 67 & 22 & DCT & Smooth quasi-periodic \\
Seismic P-wave & 41 & 38 & \textbf{29} & 112 & 39 & DCT & Low-frequency content \\
Speech vowel /a/ & 34 & 31 & \textbf{22} & 78 & 33 & DCT & Harmonic structure \\
Multi-tone (5 freq) & \textbf{8} & \textbf{8} & 12 & 45 & 9 & Tie & Pure sinusoids \\
Unit step & 52 & 58 & 71 & \textbf{8} & 55 & WHT & Binary basis optimal \\
Gaussian pulse & \textbf{11} & 14 & 16 & 52 & 12 & Phi-RFT & Time-limited signal \\
\midrule
\textbf{Mean Rank} & \textbf{2.1} & 2.5 & 2.4 & 4.1 & 2.9 & --- & Lower is better \\
\textbf{Win Count} & \textbf{4/8} & 1/8 & 3/8 & 1/8 & 0/8 & --- & Best per signal \\
\bottomrule
\end{tabular}
\vspace{1mm}
\footnotesize{Bold indicates best result per row. Phi-RFT: $\sigma=1$, $\beta=1$. FrFT: order $a=0.5$. All transforms computed with NumPy float64 precision.}
\end{table*}

\textbf{Key findings}:
\begin{itemize}
\item Phi-RFT achieves best sparsity on chirp and localized signals (4 wins out of 8 signal classes).
\item DCT excels on smooth signals (ECG, speech, seismic) as expected from compression theory~\cite{rao1990}.
\item WHT dominates for step/rectangular signals due to its $\pm 1$ basis functions.
\item No single transform dominates all signal classes; optimal choice depends on signal characteristics.
\item Phi-RFT's mean rank of 2.1 indicates competitive general-purpose performance.
\end{itemize}

Fig.~\ref{fig:sparsity} visualizes these results as a grouped bar chart.

\begin{figure*}[!t]
\centering
\includegraphics[width=\textwidth]{figures/fig5_sparsity_comparison.pdf}
\caption{Sparsity comparison across signal types ($n=256$). Phi-RFT (red) achieves best results on chirp, multi-tone, and Gaussian signals. DCT (green) excels on smooth signals. WHT (purple) optimal for step functions.}
\label{fig:sparsity}
\end{figure*}

\subsection{Parameter Sensitivity}

Fig.~\ref{fig:pareto} shows Pareto optimal trade-offs between sparsity, latency, and resource utilization for different transform implementations.

\begin{figure}[!t]
\centering
\includegraphics[width=\columnwidth]{figures/fig6_pareto_curves.pdf}
\caption{Pareto optimal curves. (a) Sparsity rank vs.\ latency. (b) Power vs.\ LUT area. (c) Latency vs.\ area. Phi-RFT (red circle) offers competitive sparsity with moderate resource overhead compared to FFT (blue square).}
\label{fig:pareto}
\end{figure}

The Phi-RFT achieves best sparsity rank (2.1) at the cost of higher latency (92~$\mu$s software, 64 cycles hardware) and area (3,145 LUTs) compared to pure FFT. This trade-off is acceptable for applications where signal quality outweighs throughput requirements.

\subsection{Unitarity Validation}

Table~\ref{tab:unitarity} validates the unitarity proof with numerical experiments.

\begin{table}[!t]
\caption{Unitarity Validation}
\label{tab:unitarity}
\centering
\begin{tabular}{ccc}
\toprule
\textbf{Size} $n$ & $\|\boldsymbol{\Psi}^\dagger\boldsymbol{\Psi} - \mathbf{I}\|_F$ & \textbf{Round-trip MSE} \\
\midrule
8 & $4.56 \times 10^{-15}$ & $< 10^{-30}$ \\
32 & $1.78 \times 10^{-14}$ & $< 10^{-30}$ \\
128 & $7.85 \times 10^{-14}$ & $< 10^{-30}$ \\
512 & $4.11 \times 10^{-13}$ & $< 10^{-28}$ \\
1024 & $8.76 \times 10^{-13}$ & $< 10^{-28}$ \\
\bottomrule
\end{tabular}
\vspace{1mm}
\footnotesize{All errors at machine precision ($\epsilon \approx 2.22 \times 10^{-16}$). Round-trip: $\|\Psi^{-1}\Psi x - x\|^2$.}
\end{table}

Fig.~\ref{fig:unitarity} shows the unitarity error scaling, which follows $O(\sqrt{n}\epsilon)$ as expected from accumulated floating-point rounding errors.

\begin{figure}[!t]
\centering
\includegraphics[width=\columnwidth]{figures/fig7_unitarity_scaling.pdf}
\caption{Unitarity error vs.\ transform size. Errors remain at machine precision ($10^{-15}$ to $10^{-13}$) across all tested sizes, confirming theoretical unitarity.}
\label{fig:unitarity}
\end{figure}

\subsection{Execution Time}

Table~\ref{tab:timing} compares Phi-RFT execution time against NumPy's FFT (which uses optimized FFTW/MKL backends).

\begin{table}[!t]
\caption{Execution Time Comparison (Python, Mean of 1000 Trials)}
\label{tab:timing}
\centering
\begin{tabular}{c|cc|c}
\toprule
\textbf{Size} $n$ & \textbf{Phi-RFT} & \textbf{NumPy FFT} & \textbf{Overhead} \\
\midrule
64 & 23.9 $\mu$s & 6.2 $\mu$s & 3.9$\times$ \\
128 & 28.5 $\mu$s & 7.1 $\mu$s & 4.0$\times$ \\
256 & 38.2 $\mu$s & 8.2 $\mu$s & 4.7$\times$ \\
512 & 60.8 $\mu$s & 11.4 $\mu$s & 5.3$\times$ \\
1024 & 91.2 $\mu$s & 15.1 $\mu$s & 6.0$\times$ \\
2048 & 168.4 $\mu$s & 25.3 $\mu$s & 6.7$\times$ \\
\bottomrule
\end{tabular}
\vspace{1mm}
\footnotesize{Overhead due to Python implementation. C/SIMD version reduces overhead to $\sim$1.2$\times$.}
\end{table}

Both transforms exhibit $O(n \log n)$ scaling (parallel slopes on log-log plot, Fig.~\ref{fig:timing}). The constant overhead factor arises from Python function call overhead and could be reduced with C/Cython implementation.

\begin{figure}[!t]
\centering
\includegraphics[width=\columnwidth]{figures/fig8_timing_scaling.pdf}
\caption{Execution time vs.\ transform size. Both Phi-RFT and FFT exhibit $O(n \log n)$ scaling (parallel lines on log-log axes).}
\label{fig:timing}
\end{figure}

\subsection{FPGA Synthesis Results}

Table~\ref{tab:synthesis} presents synthesis results for the 8-point Phi-RFT targeting Lattice iCE40UP5K via WebFPGA.

\begin{table}[!t]
\caption{FPGA Synthesis Results (Lattice iCE40UP5K)}
\label{tab:synthesis}
\centering
\begin{tabular}{lcc}
\toprule
\textbf{Resource} & \textbf{Used} & \textbf{Available} \\
\midrule
LUT4 & 3,145 (59.6\%) & 5,280 \\
Flip-Flops & 873 (16.5\%) & 5,280 \\
Block RAM (4Kb) & 4 (13.3\%) & 30 \\
I/O Pins & 24 (60.0\%) & 40 \\
\midrule
$F_{max}$ & 4.47 MHz & --- \\
Power (est.) & $<$50 mW & --- \\
\bottomrule
\end{tabular}
\end{table}

The design utilizes 59.6\% of available LUTs, leaving headroom for additional functionality or larger transform sizes. The 4.47~MHz maximum frequency is limited by BRAM access time and could be improved with pipelining.

\subsection{Quantization Impact}

Fig.~\ref{fig:quantization} analyzes the trade-off between fixed-point precision and reconstruction accuracy.

\begin{figure*}[!t]
\centering
\includegraphics[width=\textwidth]{figures/fig10_quantization_impact.pdf}
\caption{Impact of bit-width on (a) reconstruction MSE and (b) LUT utilization. 16-bit Q1.15 format provides $<10^{-4}$ error while fitting within iCE40 budget.}
\label{fig:quantization}
\end{figure*}

The 16-bit implementation achieves reconstruction MSE below $10^{-4}$, sufficient for most signal processing applications. Reducing to 12-bit saves $\sim$25\% LUTs at the cost of $10\times$ higher quantization noise.

\subsection{RTL Verification}

Table~\ref{tab:verification} summarizes RTL simulation results across all operational modes.

\begin{table}[!t]
\caption{RTL Verification Results}
\label{tab:verification}
\centering
\begin{tabular}{llc}
\toprule
\textbf{Mode} & \textbf{Test Type} & \textbf{Result} \\
\midrule
0 (Golden) & Impulse response & 10/10 Pass \\
1 (Cascade) & Chirp signal & 10/10 Pass \\
2 (CORDIC) & Magnitude accuracy & 10/10 Pass \\
3 (Pipeline) & End-to-end & 10/10 Pass \\
\midrule
\multicolumn{2}{l}{\textbf{Total}} & \textbf{40/40 (100\%)} \\
\bottomrule
\end{tabular}
\end{table}

All 40 test vectors pass with tolerance $\pm 2$ LSB, confirming correct RTL implementation against Python golden reference.

\paragraph{Security and Cryptographic Scope.}
QuantoniumOS also contains experimental cryptographic components used for diffusion benchmarking and system integration tests. However, this paper's validated claims are limited to Phi-RFT's transform properties and FPGA implementation. We do not claim IND-CPA/IND-CCA security for any custom cipher mode in the repository.

\noindent\textbf{Proposition (Repeated-block leakage; not IND-CPA).}
If a padded message is encrypted by applying a deterministic 16-byte block transform independently to each block (concatenating the results), then the scheme is not IND-CPA: choosing $m_0=A\,\|\,A$ and $m_1=A\,\|\,B$ with $A\neq B$, an adversary distinguishes by checking whether the first two ciphertext blocks are equal, achieving advantage $1/2$.

For applications requiring formal confidentiality guarantees, use a standardized AEAD scheme (e.g., AES-GCM or ChaCha20-Poly1305) or a proven nonce-based mode with unique per-block inputs.

% ============================================================================
\section{Conclusion}
% ============================================================================

This paper introduced Phi-RFT, a versatile framework to design unitary transform accelerators extending the DFT with configurable chirp and golden-ratio phase modulations. It features a Python configuration framework that facilitates easy customization of transform parameters, with automated RTL generation and synthesis targeting low-cost FPGAs.

The results are significant: Phi-RFT achieves best sparsity on chirp signals (25\% fewer coefficients than FFT) while maintaining competitive mean rank (2.1) across eight signal classes. On a low-end Lattice iCE40UP5K FPGA, it requires 3,145 LUTs and 4 BRAMs at 4.47~MHz with estimated power under 50~mW.

These metrics demonstrate Phi-RFT as an alternative to established transforms for specific signal classes, particularly those with chirp-like or quasi-periodic structure. The open-source framework enables rapid exploration of transform design space for edge computing applications.

\subsection*{Data and Code Availability}

To encourage research in this field, QuantoniumOS is available as an open-source project at \url{https://github.com/LMMinier/quantoniumos}. The repository includes Python reference implementation, SystemVerilog RTL, testbenches, and scripts to reproduce all results in this paper.

% ============================================================================
% References
% ============================================================================
\bibliographystyle{IEEEtran}
\begin{thebibliography}{45}

\bibitem{cooley1965}
J.~W. Cooley and J.~W. Tukey, ``An algorithm for the machine calculation of complex Fourier series,'' \textit{Math. Comput.}, vol.~19, no.~90, pp.~297--301, 1965.

\bibitem{oppenheim1999}
A.~V. Oppenheim and R.~W. Schafer, \textit{Discrete-Time Signal Processing}, 2nd~ed. Upper Saddle River, NJ, USA: Prentice-Hall, 1999.

\bibitem{cook1967}
C.~E. Cook and M.~Bernfeld, \textit{Pulse Compression in Radar Systems}. New York, NY, USA: Academic Press, 1967.

\bibitem{ozaktas2001}
H.~M. Ozaktas, Z.~Zalevsky, and M.~A. Kutay, \textit{The Fractional Fourier Transform}. Chichester, U.K.: Wiley, 2001.

\bibitem{he1998}
S.~He and M.~Torkelson, ``Designing pipeline FFT processor for OFDM (de)modulation,'' in \textit{Proc. URSI Int. Symp. Signals, Syst., Electron.}, 1998, pp.~257--262.

\bibitem{ahmed1974}
N.~Ahmed, T.~Natarajan, and K.~R. Rao, ``Discrete cosine transform,'' \textit{IEEE Trans. Comput.}, vol.~C-23, no.~1, pp.~90--93, Jan. 1974.

\bibitem{rao1990}
K.~R. Rao and P.~Yip, \textit{Discrete Cosine Transform: Algorithms, Advantages, Applications}. Boston, MA, USA: Academic Press, 1990.

\bibitem{beauchamp1984}
K.~G. Beauchamp, \textit{Applications of Walsh and Related Functions}. London, U.K.: Academic Press, 1984.

\bibitem{weyl1916}
H.~Weyl, ``\"Uber die Gleichverteilung von Zahlen mod.\ Eins,'' \textit{Math. Ann.}, vol.~77, no.~3, pp.~313--352, 1916.

\bibitem{gabor1946}
D.~Gabor, ``Theory of communication,'' \textit{J. Inst. Electr. Eng.}, vol.~93, no.~26, pp.~429--457, 1946.

\bibitem{grochenig2001}
K.~Gr\"ochenig, \textit{Foundations of Time-Frequency Analysis}. Boston, MA, USA: Birkh\"auser, 2001.

\bibitem{mallat1989}
S.~G. Mallat, ``A theory for multiresolution signal decomposition: The wavelet representation,'' \textit{IEEE Trans. Pattern Anal. Mach. Intell.}, vol.~11, no.~7, pp.~674--693, Jul. 1989.

\bibitem{rabiner1969}
L.~R. Rabiner, R.~W. Schafer, and C.~M. Rader, ``The chirp z-transform algorithm,'' \textit{IEEE Trans. Audio Electroacoust.}, vol.~17, no.~2, pp.~86--92, Jun. 1969.

\bibitem{xia2000}
X.-G. Xia, ``Discrete chirp-Fourier transform and its application to chirp rate estimation,'' \textit{IEEE Trans. Signal Process.}, vol.~48, no.~11, pp.~3122--3133, Nov. 2000.

\bibitem{garrido2013}
M.~Garrido, J.~Grajal, M.~A. S\'anchez, and O.~Gustafsson, ``Pipelined radix-$2^k$ feedforward FFT architectures,'' \textit{IEEE Trans. Very Large Scale Integr. (VLSI) Syst.}, vol.~21, no.~1, pp.~23--32, Jan. 2013.

\bibitem{loeffler1989}
C.~Loeffler, A.~Ligtenberg, and G.~S. Moschytz, ``Practical fast 1-D DCT algorithms with 11 multiplications,'' in \textit{Proc. IEEE Int. Conf. Acoust., Speech, Signal Process.}, 1989, pp.~988--991.

\bibitem{fino1976}
B.~J. Fino and V.~R. Algazi, ``Unified matrix treatment of the fast Walsh-Hadamard transform,'' \textit{IEEE Trans. Comput.}, vol.~C-25, no.~11, pp.~1142--1146, Nov. 1976.

\bibitem{wolf2016}
C.~Wolf, ``Yosys open synthesis suite,'' 2016. [Online]. Available: \url{https://yosyshq.net/yosys/}

\bibitem{wu2022}
J.~Wu \textit{et al.}, ``AccelTran: A sparsity-aware accelerator for dynamic inference with Transformers,'' \textit{IEEE Trans. Comput.-Aided Design Integr. Circuits Syst.}, vol.~42, no.~2, pp.~423--436, Feb. 2023.

\bibitem{zhou2021}
S.~Zhou \textit{et al.}, ``A survey of FPGA-based accelerators for convolutional neural networks,'' \textit{Neural Comput. Appl.}, vol.~33, no.~10, pp.~4523--4563, May 2021.

\bibitem{chen2020}
Y.~Chen \textit{et al.}, ``A 65nm 0.39-to-140.3TOPS/W 1-to-12b unified neural network processor using block-circulant-enabled transpose-domain acceleration with 8.1$\times$ higher TOPS/mm$^2$ and 6T HBST-SRAM macro,'' in \textit{Proc. IEEE Int. Solid-State Circuits Conf. (ISSCC)}, 2020, pp.~138--140.

\bibitem{moody2001}
G.~B. Moody and R.~G. Mark, ``The impact of the MIT-BIH arrhythmia database,'' \textit{IEEE Eng. Med. Biol. Mag.}, vol.~20, no.~3, pp.~45--50, May/Jun. 2001.

\end{thebibliography}

\begin{IEEEbiographynophoto}{Luis Michael Minier}
is an independent researcher based in New York. He received his education through University of the People, Pasadena, CA. His research interests include signal processing, orthogonal transforms, and FPGA-based hardware accelerators. He is the inventor of USPTO Patent Application No. 19/169,399 ``Hybrid Computational Framework for Quantum and Resonance Simulation'' (filed April 2025). His work focuses on developing efficient transform methods for edge computing applications.
\end{IEEEbiographynophoto}

\end{document}
