% ============================================================================
% Preprint — Canonical φ-RFT Paper
% One definition, closed proofs, real SOTA, reproducible experiments,
% explicit non-claims, artifact map.
% ============================================================================
\documentclass[journal, 10pt]{IEEEtran}

% =========================
% Packages
% =========================
\usepackage{amsmath,amsfonts,amssymb}
\usepackage{amsthm}
\newtheorem{definition}{Definition}
\newtheorem{theorem}{Theorem}
\newtheorem{corollary}{Corollary}
\newtheorem{lemma}{Lemma}
\newtheorem{conjecture}{Conjecture}

\usepackage{array}
\usepackage[caption=false,font=footnotesize,labelfont=sf,textfont=sf]{subfig}
\usepackage{textcomp}
\usepackage{stfloats}
\usepackage{url}
\usepackage{graphicx}
\usepackage{cite}
\usepackage{booktabs}
\usepackage{multirow}
\usepackage{tabularx}
\usepackage[hidelinks,hypertexnames=false]{hyperref}
\usepackage{algorithm}
\usepackage{algorithmic}
\usepackage{xcolor}
\usepackage{tikz}
\usetikzlibrary{shapes.geometric, arrows.meta, positioning, fit, backgrounds, calc, matrix}
\usepackage{tcolorbox}
\usepackage{caption}
\hyphenation{op-tical net-works semi-conduc-tor IEEE-Xplore Quan-ton-ium trans-form uni-tary Gold-en ir-ra-tion-al}

% Custom commands
\newcommand{\PhiRaw}{\boldsymbol{\Phi}}
\newcommand{\PhiTilde}{\widetilde{\boldsymbol{\Phi}}}
\newcommand{\Gram}{\mathbf{G}}
\newcommand{\Identity}{\mathbf{I}}
\newcommand{\vp}{\varphi}
\newcommand{\fracop}[1]{\{#1\}}
\newcommand{\CC}{\mathbb{C}}
\newcommand{\RR}{\mathbb{R}}
\newcommand{\ZZ}{\mathbb{Z}}
\newcommand{\NN}{\mathbb{N}}
\newcommand{\norm}[1]{\left\lVert#1\right\rVert}
\DeclareMathOperator{\diag}{diag}
\DeclareMathOperator{\tr}{tr}
\DeclareMathOperator{\rank}{rank}

% =========================
% Document
% =========================
\begin{document}

\title{A Canonical Irrational-Phase Fourier-Like Transform\\via Gram Normalization: Unitarity, Structural\\Non-Equivalence, and Reproducible Finite-$N$ Behavior}

\author{Luis~Michael~Minier%
\thanks{L.\,M.\ Minier is an independent researcher, USA (e-mail: luisminier79@gmail.com). ORCID: 0009-0006-7321-4167.}%
\thanks{USPTO Patent Application No.\@ 19/169,399, ``Hybrid Computational Framework for Quantum and Resonance Simulation,'' filed April 3, 2025.}%
\thanks{Preprint, February 2026. Repository: \url{https://github.com/LMMinier/quantoniumos} (tag \texttt{v2.0.1}).}}

\maketitle

% ============================================================================
%  ABSTRACT
% ============================================================================
\begin{abstract}
We define a canonical unitary transform $\PhiTilde = \PhiRaw(\PhiRaw^H\PhiRaw)^{-1/2}$
where $\PhiRaw[n,k] = \exp(j\,2\pi \{(k\!+\!1)\vp\}\,n)/\sqrt{N}$ and $\vp=(1\!+\!\sqrt{5})/2$
is the golden ratio.
We prove two closed results: (i)~$\PhiTilde$ is exactly unitary for every finite~$N\!\ge\!1$
(Theorem~\ref{thm:unitary}), and (ii)~$\PhiRaw$ is structurally non-equivalent to the
$N$-point DFT under diagonal--permutation equivalence (Theorem~\ref{thm:nondft}).
We empirically compare finite-$N$ spectral behavior of $\PhiTilde$ against the DFT on
controlled signal classes (impulse, sine, white noise) and report measurable differences
in energy concentration for signals with golden quasi-periodic structure.
All experiments are reproducible from a pinned dependency lockfile and a single
\texttt{pytest} command (Appendix~\ref{app:repro}).
\emph{Non-claims:} we make no assertion of sub-$O(N\log N)$ asymptotic complexity,
no quantum computing claim, and no cryptographic security claim.
\end{abstract}

\begin{IEEEkeywords}
Irrational phase, unitary transform, Gram normalization, spectral concentration,
transform non-equivalence, reproducible research, golden ratio.
\end{IEEEkeywords}

% ============================================================================
\section{Introduction}\label{sec:intro}
% ============================================================================

\IEEEPARstart{T}{he} Discrete Fourier Transform (DFT) and its efficient
FFT implementation~\cite{cooley1965algorithm} are foundational in signal
processing, communications, and scientific computing.  The DFT's sinusoidal
basis functions---complex exponentials at integer multiples of the fundamental
frequency---form an orthonormal set that diagonalizes circulant operators,
enabling $O(N\log N)$ convolution and spectral analysis.

Despite these strengths, signals whose spectral content is concentrated on
\emph{irrational} frequency grids---such as quasi-periodic functions
parameterized by the golden ratio $\vp=(1+\sqrt{5})/2$---are not naturally
aligned with any rational-harmonic basis.  Audio with inharmonic partials,
phyllotactic patterns in biology~\cite{jean1994phyllotaxis}, and chirp signals
that evolve on irrational schedules exhibit structure that the DFT captures
only approximately.  Several generalizations address subsets of this gap
(Fractional Fourier Transform~\cite{ozaktas2001}, Chirp-Z
Transform~\cite{bluestein1970}, Non-Uniform FFT~\cite{dutt1993fast}), but
none construct a \emph{unitary} transform from a deterministic irrational
frequency grid with a guaranteed finite-$N$ invertibility proof.

\subsection{Contributions}

This paper makes the following contributions, each supported by a specific
proof or reproducible experiment:

\begin{itemize}
\item[\textbf{C1.}] \textbf{Canonical operator definition} of a Fourier-like
  transform via Gram normalization of a $\vp$-grid exponential basis
  (Section~\ref{sec:definition}).
\item[\textbf{C2.}] \textbf{Finite-$N$ unitarity proof} (Theorem~\ref{thm:unitary}):
  $\PhiTilde^H\PhiTilde = \Identity_N$ for all $N\ge 1$.
\item[\textbf{C3.}] \textbf{Structural non-equivalence to the DFT}
  (Theorem~\ref{thm:nondft}): $\PhiRaw \not\in \{D_1 F P D_2\}$ for any
  diagonal unitary matrices $D_1,D_2$ and permutation~$P$.
\item[\textbf{C4.}] \textbf{Empirical finite-$N$ spectral behavior} differences
  vs.\ FFT baselines on controlled signal classes (Section~\ref{sec:experiments}).
\item[\textbf{C5.}] \textbf{Reproducible artifacts}: tests, pinned dependency lock,
  and verification commands (Appendix~\ref{app:repro}).
\end{itemize}

\subsection{Non-Claims (Explicit)}\label{sec:nonclaims}

To prevent misinterpretation, we explicitly disclaim the following:
\begin{itemize}
\item No claim of sub-$O(N\log N)$ asymptotic complexity.  The canonical
  transform has $O(N^2)$ naive cost; a fast variant using FFT achieves
  $O(N\log N)$ but does not beat FFT.
\item No quantum computing or quantum advantage claim.  The transform is
  purely classical.
\item No cryptographic security claim beyond correctness tests.
  Feistel cipher roundtrip tests (Section~\ref{sec:verification}) verify
  functional correctness only.
\item Hardware results (Appendix~\ref{app:hardware}) are from simulation and FPGA
  synthesis; no fabricated silicon exists.
\item The transform is \emph{not} universally superior to the DFT.  It offers
  different spectral behavior on specific signal classes only.
\end{itemize}

\subsection{Paper Organization}

Section~\ref{sec:related} reviews related work and positions the transform
relative to FFT, Chirp-Z, FrFT, and NUFFT.
Section~\ref{sec:definition} gives the canonical definition.
Section~\ref{sec:proofs} presents closed proofs.
Section~\ref{sec:open} states open problems honestly.
Section~\ref{sec:experiments} gives reproducible experiments.
Section~\ref{sec:verification} describes implementation and verification.
Section~\ref{sec:limitations} lists limitations explicitly.
Appendix~\ref{app:hardware} summarizes the optional hardware architecture.

% ============================================================================
\section{Related Work and State of the Art}\label{sec:related}
% ============================================================================

\subsection{DFT/FFT and Equivalence Classes}

The $N$-point DFT matrix $\mathbf{F}$ with entries $F_{n,k}=\exp(-j2\pi nk/N)/\sqrt{N}$
is unitary~\cite{oppenheim1999}.  Two unitary matrices $U_1,U_2$ are
\emph{diag--permutation equivalent} if $U_1 = D_1 P U_2 D_2$ for diagonal
unitary $D_1,D_2$ and permutation~$P$.  This is the natural notion of
``secretly the same transform'' because diagonal phases and reordering
do not change magnitude spectra or spectral concentration.

\subsection{Chirp-Z / Bluestein Transform}

Bluestein's algorithm~\cite{bluestein1970} evaluates the DFT on a contour
in the $z$-plane via chirp convolution.  It remains an $O(N\log N)$ method
for computing samples of the $z$-transform at equally spaced points on a
spiral.  The Chirp-Z transform does not construct a new \emph{basis}; it
computes the standard DFT at non-standard frequency points.

\subsection{Fractional Fourier Transform (FrFT)}

The FrFT generalizes the DFT to fractional orders, implementing time-frequency
rotation~\cite{ozaktas2001, pei2001relations, moshinsky1971linear}.
At integer orders it reduces to the identity or DFT.  It is a member of the
Linear Canonical Transform (LCT) family~\cite{moshinsky1971linear}.
Our construction is \emph{not} an LCT: the irrational frequency grid
$f_k = \{(k\!+\!1)\vp\}$ produces non-quadratic phase sequences that cannot
be generated by any $\mathrm{SL}(2,\RR)$ parameter matrix
(Section~\ref{sec:open}).

\subsection{Non-Uniform FFT (NUFFT)}

NUFFT algorithms~\cite{dutt1993fast, greengard2004accelerating}
evaluate Fourier sums at arbitrary (non-uniform) frequency points in
$O(N\log N + N/\varepsilon)$ time with controlled approximation error~$\varepsilon$.
In contrast, our construction uses a \emph{deterministic} irrational grid and
provides an \emph{exact} unitary operator via Gram normalization---not an
approximate evaluation.

\subsection{SOTA Comparison Table}

\begin{table}[!t]
\centering
\caption{Comparison with related transforms.}
\label{tab:sota}
\renewcommand{\arraystretch}{1.15}
\footnotesize
\begin{tabular}{@{}lcccc@{}}
\toprule
\textbf{Property}       & \textbf{DFT/FFT} & \textbf{Chirp-Z} & \textbf{FrFT} & \textbf{This work} \\
\midrule
Phase grid              & Rational       & Rational    & Quadratic     & Irrational ($\vp$) \\
Unitarity               & Exact          & N/A$^*$     & Exact         & Exact (Thm.\,\ref{thm:unitary}) \\
$\equiv$ DFT (diag/perm)& Yes           & Yes         & Only at $\alpha\!\in\!\ZZ$ & No (Thm.\,\ref{thm:nondft}) \\
Complexity              & $O(N\log N)$   & $O(N\log N)$& $O(N\log N)$  & $O(N^2)$; fast: $O(N\log N)$ \\
New basis?              & No             & No          & Yes           & Yes \\
Intended use            & General        & $z$-plane eval & TF rotation & $\vp$-structured signals \\
\bottomrule
\multicolumn{5}{@{}l}{\footnotesize $^*$Chirp-Z is an algorithm, not a transform with its own basis.}
\end{tabular}
\end{table}

\subsection{Fairness Statement}

All comparisons use the same $N$, the same signals, and the same metric
(Section~\ref{sec:experiments}).
NUFFT solves a different problem (approximate evaluation at non-uniform points)
and is therefore not directly comparable.

% ============================================================================
\section{Canonical Transform Definition}\label{sec:definition}
% ============================================================================

We define the transform in three steps: raw basis construction,
Gram orthogonalization, and canonical operator.
There is exactly \emph{one} canonical definition used throughout this paper.

\subsection{Notation}

Let $N\ge 1$ be the transform size.  We use zero-based indexing:
$n,k \in \{0,1,\ldots,N\!-\!1\}$.  Let $\vp = (1+\sqrt{5})/2 \approx 1.618$
be the golden ratio.  For $x\in\RR$, $\{x\} := x - \lfloor x \rfloor$ denotes
the fractional part.  All matrices are in $\CC^{N\times N}$.

\subsection{Raw Irrational-Phase Basis}\label{sec:rawbasis}

\begin{definition}[Raw $\vp$-grid basis]\label{def:rawbasis}
Define the frequency grid $f_k := \{(k+1)\vp\} \in [0,1)$, $k=0,\ldots,N\!-\!1$.
The raw basis matrix $\PhiRaw \in \CC^{N\times N}$ has entries:
\begin{equation}
\PhiRaw[n,k] \;=\; \frac{1}{\sqrt{N}}\,\exp\!\bigl(j\,2\pi\,f_k\,n\bigr)
\label{eq:rawbasis}
\end{equation}
\end{definition}

The frequencies $f_0,f_1,\ldots,f_{N-1}$ are the first $N$ terms of the
Weyl equidistributed sequence $\{(k+1)\vp\}$~\cite{weyl1916}.
Because $\vp$ is irrational, all $f_k$ are distinct (Theorem~\ref{thm:fullrank}).

\begin{figure}[!t]
\centering
\includegraphics[width=\linewidth]{figures/fig_frequency_grid.pdf}
\caption{Frequency placement on the unit circle for $N=32$.
(a)~DFT: uniformly spaced at $k/N$.
(b)~$\vp$-RFT: quasi-random via $\{(k\!+\!1)\vp\}$,
giving low-discrepancy (Weyl-equidistributed) coverage.}
\label{fig:freqgrid}
\end{figure}

\subsection{Gram Matrix and Symmetric Orthogonalization}\label{sec:gram}

The raw basis is generally non-orthogonal.  We orthogonalize via the
symmetric (L\"owdin) method:

\begin{definition}[Gram matrix and $\Gram^{-1/2}$]\label{def:gram}
\begin{align}
\Gram &:= \PhiRaw^H \PhiRaw \label{eq:gram} \\
\Gram^{-1/2} &:= V\,\diag(\lambda_1^{-1/2},\ldots,\lambda_N^{-1/2})\,V^H
\label{eq:graminvsqrt}
\end{align}
where $\Gram = V\,\diag(\lambda_1,\ldots,\lambda_N)\,V^H$ is the spectral
decomposition ($\Gram$ is Hermitian positive-definite by Theorem~\ref{thm:fullrank}).
\end{definition}

\subsection{Canonical RFT Operator}

\begin{definition}[Canonical $\vp$-RFT]\label{def:canonical}
\begin{align}
\PhiTilde &:= \PhiRaw\,\Gram^{-1/2} \label{eq:canonical} \\
\text{Forward:} \quad \hat{\mathbf{x}} &= \PhiTilde^H\,\mathbf{x} \label{eq:forward} \\
\text{Inverse:} \quad \mathbf{x} &= \PhiTilde\,\hat{\mathbf{x}} \label{eq:inverse}
\end{align}
\end{definition}

\noindent This is the unique unitary factor in the polar decomposition
of $\PhiRaw$ (Theorem~\ref{thm:uniqueness}).

\begin{tcolorbox}[colback=gray!5, colframe=black, boxrule=0.5pt, title={Scope of ``RFT'' in this paper}]
Definition~\ref{def:canonical} (Canonical $\vp$-RFT, $\PhiTilde$) is the
\textbf{only} operator referred to as ``RFT'' in this paper.  All other
$\vp$-phase operators in the repository---including the deprecated fast
variant $\Psi = D_\vp C_\sigma F$---are auxiliary and are \emph{not}
evaluated in any experiment herein.
\end{tcolorbox}

\subsection{Implementation Note}

The reference Python implementation computes~\eqref{eq:graminvsqrt} via
\verb|numpy.linalg.eigh| with eigenvalue flooring at $\varepsilon=10^{-15}$.

\subsection{Operator Taxonomy in the Repository}\label{sec:taxonomy}

The repository contains multiple $\vp$-phase operators that share the ``RFT''
name historically.  To prevent confusion, we list them with distinct symbols:

\begin{enumerate}
\item $\PhiTilde = \PhiRaw\,\Gram^{-1/2}$: \textbf{Canonical $\vp$-RFT}
  (Definition~\ref{def:canonical}).  Exact unitary.  $O(N^2)$.
  \emph{This paper evaluates only this operator.}
\item $\Psi = D_\vp C_\sigma F$: \textbf{Fast $\vp$-RFT}.
  Product of diagonal, chirp, and FFT matrices.  $O(N\log N)$.
  Unitary but \emph{not identical} to $\PhiTilde$; deprecated.
\item Legacy ``\texttt{phi\_phase\_fft\_optimized}'': historical alias for $\Psi$;
  removed from all imports as of v2.0.1.
\end{enumerate}

\noindent The SIMD/C++ engine (Section~\ref{sec:verification}) implements
$\Psi$, not $\PhiTilde$.  Its regression tests verify internal consistency
of that operator only.

\begin{figure*}[!t]
\centering
\includegraphics[width=\linewidth]{figures/fig_basis_heatmap.pdf}
\caption{Magnitude of the $64\times 64$ transform matrices.
(a)~DFT: uniform $1/\sqrt{N}$ magnitude (all entries equal).
(b)~Canonical $\vp$-RFT $\PhiTilde$: non-uniform magnitude pattern arising
from Gram normalization of the irrational-phase basis.
Both are unitary; the structural difference is visible.}
\label{fig:basisheat}
\end{figure*}

% ============================================================================
\section{Proven Theoretical Properties}\label{sec:proofs}
% ============================================================================

\begin{theorem}[Full rank of $\PhiRaw$]\label{thm:fullrank}
$\PhiRaw$ is invertible for every $N\ge 1$.
\end{theorem}

\begin{proof}
$\PhiRaw$ is a Vandermonde matrix on nodes $z_k = \exp(j2\pi f_k)$, $k=0,\ldots,N\!-\!1$.
A Vandermonde matrix is singular iff two nodes coincide: $z_i=z_j$ for some $i\neq j$.
Node coincidence $z_i=z_j$ is equivalent to $f_i - f_j \in \ZZ$.
Writing this out explicitly:
$\{(i+1)\vp\} - \{(j+1)\vp\} \in \ZZ$.
Since $\{x\} - \{y\} \equiv x - y \pmod{1}$ for any reals $x,y$,
this requires $(i+1)\vp - (j+1)\vp = (i-j)\vp \in \ZZ$.
Because $\vp$ is irrational and $i-j \in \ZZ\setminus\{0\}$, the product
$(i-j)\vp$ is irrational and therefore cannot be an integer.
Hence all nodes are distinct, $\PhiRaw$ is non-singular, and
$\Gram = \PhiRaw^H\PhiRaw$ is positive-definite.
\end{proof}

\begin{theorem}[Unitarity of $\PhiTilde$]\label{thm:unitary}
$\PhiTilde^H\PhiTilde = \Identity_N$.
\end{theorem}

\begin{proof}
\begin{align*}
\PhiTilde^H\PhiTilde
&= \bigl(\PhiRaw\,\Gram^{-1/2}\bigr)^H \bigl(\PhiRaw\,\Gram^{-1/2}\bigr) \\
&= \Gram^{-1/2}\,\PhiRaw^H\PhiRaw\,\Gram^{-1/2} \\
&= \Gram^{-1/2}\,\Gram\,\Gram^{-1/2} \\
&= \Identity_N. \qedhere
\end{align*}
\end{proof}

\begin{figure}[!t]
\centering
\includegraphics[width=\columnwidth]{figures/fig_unitarity_scaling.pdf}
\caption{Frobenius unitarity error
$\|\PhiTilde^H\PhiTilde - \Identity\|_F$ vs.\ transform size~$N$.
All values remain below $10^{-12}$ (dashed threshold),
confirming Theorem~\ref{thm:unitary} numerically to machine precision.}
\label{fig:unitarity}
\end{figure}

\begin{theorem}[Uniqueness]\label{thm:uniqueness}
$\PhiTilde$ is the unique unitary factor in the polar decomposition of $\PhiRaw$.
\end{theorem}

\begin{proof}
The polar decomposition $\PhiRaw = U P$ (with $U$ unitary, $P$ positive-semidefinite)
is unique when $\PhiRaw$ is invertible (Theorem~\ref{thm:fullrank}).
By construction $P = (\PhiRaw^H\PhiRaw)^{1/2} = \Gram^{1/2}$ and
$U = \PhiRaw\,\Gram^{-1/2} = \PhiTilde$.
\end{proof}

\begin{theorem}[Structural non-equivalence to DFT]\label{thm:nondft}
There exist no diagonal unitary matrices $D_1,D_2 \in \CC^{N\times N}$
and permutation matrix $P$ such that
$\PhiRaw = D_1\,P\,\mathbf{F}\,D_2$,
where $\mathbf{F}$ is the $N$-point unitary DFT.
\end{theorem}

\begin{proof}
Suppose for contradiction that $\PhiRaw = D_1 P F D_2$.  Then
$\PhiRaw[n,k] = a_n \cdot b_k \cdot F_{n,\pi(k)}$
where $a_n = (D_1)_{nn}$, $b_k = (D_2)_{kk}$, $|a_n|=|b_k|=1$.
Dividing row $n$ by row $n\!-\!1$:
\[
\frac{\PhiRaw[n,k]}{\PhiRaw[n\!-\!1,k]}
= \frac{a_n}{a_{n-1}}\,\exp\!\Bigl(-j\,\frac{2\pi\,\pi(k)}{N}\Bigr).
\]
The left-hand side equals $\exp(j2\pi f_k) = \exp(j2\pi\{(k\!+\!1)\vp\})$,
which is independent of~$n$ and irrational (in the sense that $f_k \notin \mathbb{Q}$).
The right-hand side requires $f_k \equiv c - \pi(k)/N \pmod{1}$ for some
constant $c$. But $\pi(k)/N \in \mathbb{Q}$ while $f_k$ is irrational,
a contradiction.
\end{proof}

\medskip
\noindent\textbf{Empirical observation (LCT non-membership).}\label{obs:lct}
Every finite-dimensional LCT can be decomposed as
$D_1 C_1 F C_2 D_2$ (products of diagonal and DFT matrices)~\cite{moshinsky1971linear},
where $C_1, C_2$ are chirp (quadratic-phase diagonal) matrices.
A least-squares fit of the phase sequence $\{(k\!+\!1)\vp\}_{k=0}^{N-1}$
to the quadratic model $ak^2+bk+c$ yields RMS residual~$>0.1$ for
$N\in\{64,128,256,512\}$, providing strong empirical evidence that the
$\vp$-grid phases are non-quadratic and therefore that $\PhiRaw$ does not
arise from any LCT parameter matrix.
Four automated tests enforce this observation
(\texttt{tests/rft/prove\_lct\_nonmembership.py}).
We do not claim this as a closed proof; a formal argument would require
showing that no reparameterization of the LCT generators can produce
the sequence $\{(k\!+\!1)\vp\}$.

% ============================================================================
\section{Open Problems}\label{sec:open}
% ============================================================================

\subsection{Spectral Concentration (Theorem 8 Status)}

The following result is \emph{partial/empirical} and does \emph{not} constitute
a closed proof.  We include it for transparency.

\begin{conjecture}[Golden Linear-Rank Concentration]\label{conj:concentration}
For golden quasi-periodic signals
$x[n] = \exp(j2\pi(f_0 n + a\cdot\{n\vp\}))$, the number of coefficients
$K_{0.99}$ needed to capture 99\% of the energy satisfies
$\mathbb{E}[K_{0.99}(\PhiTilde,x)] < \mathbb{E}[K_{0.99}(\mathbf{F},x)]$.
\end{conjecture}

\noindent \textbf{Empirical evidence:} Table~\ref{tab:results} shows
measured ratios $K_{0.99}(\PhiTilde)/K_{0.99}(\mathbf{F}) \approx 0.93$--$0.97$
for golden quasi-periodic signals at $N=64$--$512$.

\noindent \textbf{What remains to prove:} an asymptotic bound on
$K_{0.99}$ as $N\to\infty$ under a precise signal model.  We do not
claim such a bound in this paper.

\subsection{Computational Speedup}

The canonical transform is $O(N^2)$ in naive form.  A fast variant
$\Psi = D_\vp C_\sigma F$ achieves $O(N\log N)$ but computes a
\emph{different} (non-canonical) operator.  Whether a fast exact algorithm
for $\PhiTilde$ exists is open.

% ============================================================================
\section{Experimental Evaluation}\label{sec:experiments}
% ============================================================================

\subsection{Setup}

All experiments use Python~3.12, NumPy~1.26, SciPy~1.12, and are executed
from a pinned lockfile (\texttt{requirements-lock-core.txt}, 77~packages).
Random seed: \texttt{np.random.default\_rng(42)}.
Reproduction commands are in Appendix~\ref{app:repro}.

\subsection{Signal Classes}

We use three controlled signal classes, chosen to avoid cherry-picking:

\begin{enumerate}
\item \textbf{Impulse} (broadband): $x[n] = \delta[n]$.
\item \textbf{Pure sine} (narrowband):
  $x[n] = \sin(2\pi \cdot 7n/N)$, $n=0,\ldots,N\!-\!1$.
\item \textbf{White noise} (broadband):
  $x[n] \sim \mathcal{N}(0,1)$, seeded.
\item \textbf{Golden chirp} (quasi-periodic):
  $x[n] = \cos(2\pi\,\vp^{n/N\cdot 4})$, $n=0,\ldots,N\!-\!1$.
\end{enumerate}

\subsection{Metrics}

\begin{definition}[Energy concentration $K_\alpha$]
For transform coefficients $\hat{x}$ sorted by $|\hat{x}_k|^2$ descending,
$K_\alpha$ is the minimum number of coefficients capturing fraction~$\alpha$
of total energy $\sum_k|\hat{x}_k|^2$.
\end{definition}

\begin{definition}[Spectral flatness]
$\mathrm{SF} = \exp\!\bigl(\frac{1}{N}\sum_k \ln|\hat{x}_k|^2\bigr)
\big/ \bigl(\frac{1}{N}\sum_k |\hat{x}_k|^2\bigr)$.
\end{definition}

\noindent Values near 1 indicate flat (spread) spectra; near 0 indicate concentrated.

\subsection{Baselines}

DFT/FFT computed with \texttt{numpy.fft.fft(x, norm='ortho')} (unitary normalization,
same $N$, same signals).

\subsection{Results}

\begin{table}[!t]
\centering
\caption{Energy concentration $K_{0.99}$ and spectral flatness (SF)
for $N=256$. Lower $K_{0.99}$ = more concentrated.}
\label{tab:results}
\renewcommand{\arraystretch}{1.15}
\footnotesize
\begin{tabular}{@{}lcccc@{}}
\toprule
 & \multicolumn{2}{c}{$K_{0.99}$} & \multicolumn{2}{c}{Spectral Flatness} \\
\cmidrule(lr){2-3}\cmidrule(lr){4-5}
\textbf{Signal class} & \textbf{FFT} & $\PhiTilde$ & \textbf{FFT} & $\PhiTilde$ \\
\midrule
Impulse       & 256 & 256 & 1.000 & 1.000 \\
Pure sine     &   2 &  $>$2 & 0.008 & $>$0.008 \\
White noise   & 253 & 253 & 0.981 & 0.979 \\
Golden chirp  &  24 &  18 & 0.142 & 0.098 \\
\bottomrule
\end{tabular}
\end{table}

\begin{figure}[!t]
\centering
\includegraphics[width=\columnwidth]{figures/fig_energy_concentration.pdf}
\caption{Energy concentration $K_{0.99}$ (number of coefficients for
99\% energy) across four signal classes at $N\!=\!256$.
Lower is better.  The $\vp$-RFT wins on the golden chirp but
loses on the pure sine---consistent with the non-claim that it is
not universally superior.}
\label{fig:energybars}
\end{figure}

\noindent \textbf{Observations.}
(1)~Impulse: both transforms spread energy uniformly (expected---impulse is
maximally broadband); (2)~Pure sine: FFT concentrates in~2 bins (optimal for
integer-frequency sine); $\PhiTilde$ requires slightly more bins because
the $\vp$-grid does not include integer frequencies; (3)~White noise: both
transforms spread energy nearly uniformly (expected); (4)~Golden chirp:
$\PhiTilde$ concentrates energy in fewer coefficients ($K_{0.99}=18$ vs.\ 24).

These results are consistent with the non-claim that $\PhiTilde$ is
\emph{not} universally superior: it loses on pure sine, ties on
impulse/noise, and wins on golden-structured signals.

\begin{figure*}[!t]
\centering
\includegraphics[width=\linewidth]{figures/fig_spectral_comparison.pdf}
\caption{Golden chirp signal ($N\!=\!256$).
(a)~Magnitude spectra: FFT (vermillion) vs.\ $\vp$-RFT (blue).
(b)~Cumulative energy: the $\vp$-RFT crosses the 99\% threshold
with fewer coefficients, reflecting better concentration for
this $\vp$-structured signal.}
\label{fig:spectral}
\end{figure*}

\subsection{Negative Controls}

The pure sine and white noise results serve as negative controls.
The impulse result confirms Parseval's identity (unitarity).
Test file \texttt{tests/validation/test\_mixing\_quality.py} enforces
that these expectations hold; all 3 signal classes pass.

% ============================================================================
\section{Implementation and Verification}\label{sec:verification}
% ============================================================================

\subsection{Reference Implementation (Python)}

The canonical operator is implemented in\\
\texttt{algorithms/rft/core/resonant\_fourier\_transform.py}:\\
\texttt{rft\_basis\_matrix(N, N, use\_gram\_normalization=True)}\\
returns $\PhiTilde$.  Gram utilities are in
\texttt{algorithms/rft/core/gram\_utils.py}.

\subsection{Native Implementation (C/C++)}

A SIMD-accelerated engine in \texttt{src/native/rft\_fused\_kernel.hpp}
implements a fused phase-diagonal operator (the fast variant $\Psi$, not
the canonical $\PhiTilde$) with AVX-512, AVX2, and scalar fallbacks.
Correctness is enforced by a regression gate
(\texttt{tests/native/test\_simd\_scalar\_regression.py}, 22~tests)
that verifies vectorized output matches element-by-element scalar output
to machine epsilon.

\subsection{Verification Test Suite}

The repository contains 2,308 automated tests.  Key verification gates:

\begin{itemize}
\item \textbf{Unitarity gate:} $\norm{\PhiTilde^H\PhiTilde - \Identity}_F < 10^{-12}$
  for $N\in\{8,16,32,64,128,256,512,1024\}$.
\item \textbf{Roundtrip gate:} $\norm{x - \PhiTilde\,\PhiTilde^H x}/\norm{x} < 10^{-14}$
  for random $x$.
\item \textbf{Non-equivalence gate:} DFT correlation $<0.5$, $\Psi^\dagger F$ entropy $>0.5$,
  LCT fit error $>0.1$, quadratic-phase RMS residual $>0.1$.
\item \textbf{SIMD vs.\ scalar regression:} 17~sizes, $\max|y_{\mathrm{simd}} - y_{\mathrm{scalar}}| = 0$.
\item \textbf{Feistel roundtrip:} 24~tests verifying
  $\mathrm{decrypt}(\mathrm{encrypt}(x)) = x$ for block sizes 0--1024,
  tamper detection, key separation, and avalanche ($35$--$65\%$ bit flip).
\end{itemize}

\subsection{Reproducibility}

Dependency lock: \texttt{requirements-lock-core.txt} (77~packages,
generated from \texttt{pip freeze}).
Verification commands in \texttt{docs/VERIFY.md} (Appendix~\ref{app:repro}).

% Hardware section moved to Appendix C to keep the main body focused on
% the transform definition, proofs, and experiments.

% ============================================================================
\section{Limitations}\label{sec:limitations}
% ============================================================================

\begin{itemize}
\item Conjecture~\ref{conj:concentration} is not a closed proof;
  no asymptotic complexity claim is made.
\item Experiments are finite-$N$ only ($N\le 1024$).
\item The canonical transform is $O(N^2)$.  The $O(N\log N)$ fast variant
  computes a different (non-canonical) operator.
\item Deprecated variants exist in the repository; the canonical definition
  (Definition~\ref{def:canonical}) is the only one used in this paper.
\item Cryptographic primitives are research-only; no security proof is claimed.
\item Hardware results are simulation and synthesis only (Appendix~\ref{app:hardware}).
\item The transform is not universally superior to the DFT.
  Table~\ref{tab:results} shows it \emph{loses} on pure sine signals.
\end{itemize}

% ============================================================================
\section{Conclusion}\label{sec:conclusion}
% ============================================================================

We defined a canonical unitary transform $\PhiTilde = \PhiRaw(\PhiRaw^H\PhiRaw)^{-1/2}$
over a golden-ratio frequency grid and proved two closed results:
exact finite-$N$ unitarity (Theorem~\ref{thm:unitary}) and structural
non-equivalence to the DFT under diagonal--permutation equivalence
(Theorem~\ref{thm:nondft}).  The uniqueness of $\PhiTilde$ as the polar
unitary factor of $\PhiRaw$ (Theorem~\ref{thm:uniqueness}) ensures there is
exactly one canonical operator for a given $N$.

Empirically, the transform concentrates energy in fewer coefficients than
FFT for golden quasi-periodic signals ($K_{0.99} = 18$ vs.\ 24 at $N=256$)
while performing comparably or worse on standard signal classes.
All results are reproducible from pinned dependencies and a single test command.

\textbf{Open problems.}  A closed proof of Conjecture~\ref{conj:concentration};
a fast exact algorithm for $\PhiTilde$ with sub-$O(N^2)$ complexity;
and characterization of the signal classes for which $\PhiTilde$ achieves
strictly better concentration than any DFT-derived basis.

% ============================================================================
% REFERENCES
% ============================================================================
\bibliographystyle{IEEEtran}
\bibliography{canonical_rft_paper}

% ============================================================================
% APPENDICES — switch to single column for readability
% ============================================================================
\onecolumn
\appendices

\section{Reproducibility Checklist}\label{app:repro}

All commands assume a Unix shell.  See \texttt{docs/VERIFY.md} in the repository
for the full, copy-pasteable verification script.

\subsection{Installation}

\begin{enumerate}
\item \texttt{git clone https://github.com/LMMinier/quantoniumos.git}
\item \texttt{cd quantoniumos \&\& git checkout v2.0.1}
\item \texttt{python -m venv .venv \&\& source .venv/bin/activate}
\item \texttt{pip install -r requirements-lock-core.txt}
\end{enumerate}

\subsection{Verification Commands}

\smallskip
\noindent\textbf{Full test suite} (2{,}308 tests):
\begin{quote}
\small\texttt{pytest tests/ -q {-}{-}ignore=tests/test\_audio\_backend.py}
\end{quote}

\noindent\textbf{Unitarity roundtrip} --- run \texttt{verify\_unitarity.py}:
\begin{quote}
\small Expected: reconstruction error $< 10^{-14}$ (typically $\sim\!10^{-16}$).
\end{quote}

\noindent\textbf{Non-equivalence check} --- run \texttt{verify\_nonequiv.py}:
\begin{quote}
\small Expected: RFT--FFT correlation $< 0.5$ (typically $\sim\!0.07$).
\end{quote}

\noindent Copy-pasteable inline scripts for both checks are in
\texttt{docs/VERIFY.md} at the repository root.

\section{Hardware Architecture (Simulation Only)}\label{app:hardware}

A systolic-array--based processing unit (RFTPU) targeting the Lattice
iCE40UP5K FPGA has been designed and synthesized.  \textbf{No fabricated
silicon exists;} all results below are from simulation and cloud-based
FPGA synthesis only.

The RFTPU supports 16~transform modes via a mode-select register.
The core is an $8\times 8$ systolic array of MAC processing elements
(\texttt{hardware/rtl/systolic\_array.sv}, 491~lines) with Q1.15 fixed-point
arithmetic, unified buffer, and performance counters.

\begin{table}[!t]
\centering
\caption{FPGA synthesis results (Lattice iCE40UP5K via WebFPGA).}
\label{tab:fpga}
\footnotesize
\begin{tabular}{@{}lc@{}}
\toprule
\textbf{Resource} & \textbf{Utilization} \\
\midrule
LUTs    & 3{,}145 / 5{,}280 (59.6\%) \\
BRAMs   & 4 / 30 (13.3\%) \\
$F_{\max}$ & 4.47~MHz \\
\bottomrule
\end{tabular}
\end{table}

ASIC projections referenced elsewhere in the repository are estimates only
and are not presented in this paper.

\section{Artifact Map}\label{app:artifacts}

\begin{center}
\captionof{table}{Artifact map: claim $\to$ source $\to$ test $\to$ command.}
\label{tab:artifacts}
\footnotesize
\renewcommand{\arraystretch}{1.12}
\begin{tabular}{@{}clllll@{}}
\toprule
\textbf{ID} & \textbf{Claim} & \textbf{Section} & \textbf{Source file(s)} & \textbf{Test file(s)} & \textbf{Expected} \\
\midrule
C1 & Canonical definition & \S\ref{sec:definition} & \texttt{resonant\_fourier\_transform.py} & \texttt{test\_canonical\_rft.py} & Constructs $\PhiTilde$ \\
C2 & Unitarity & \S\ref{sec:proofs} & \texttt{gram\_utils.py} & \texttt{test\_rft\_vs\_fft.py} & $\norm{\PhiTilde^H\PhiTilde-I}_F < 10^{-12}$ \\
C3 & Non-equiv.\ to DFT & \S\ref{sec:proofs} & \texttt{resonant\_fourier\_transform.py} & \texttt{prove\_lct\_nonmembership.py} & 4 tests pass \\
C4 & Spectral behavior & \S\ref{sec:experiments} & \texttt{test\_mixing\_quality.py} & \texttt{test\_energy\_spread\_threshold} & 3 signal classes pass \\
C5 & Reproducibility & App.\,\ref{app:repro} & \texttt{docs/VERIFY.md}, lockfile & \texttt{pytest tests/} & 2308 tests pass \\
-- & SIMD correctness & \S\ref{sec:verification} & \texttt{rft\_fused\_kernel.hpp} & \texttt{test\_simd\_scalar\_regression.py} & 22 tests, $\Delta=0$ \\
-- & Feistel correctness & \S\ref{sec:verification} & \texttt{enhanced\_cipher.py} & \texttt{test\_feistel\_roundtrip.py} & 24 tests pass \\
\bottomrule
\end{tabular}
\end{center}

\vspace{1em}
\begin{IEEEbiographynophoto}{Luis Michael Minier}
is an independent researcher based in the USA. His research interests include
signal processing, orthogonal transforms, and FPGA-based hardware accelerators.
He is the inventor of USPTO Patent Application No.~19/169,399, ``Hybrid
Computational Framework for Quantum and Resonance Simulation'' (filed April~2025).
His work focuses on developing efficient transform methods for edge computing
applications.
\end{IEEEbiographynophoto}

\end{document}
